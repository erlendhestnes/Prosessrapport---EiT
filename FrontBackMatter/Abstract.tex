% Abstract

\pdfbookmark[1]{Sammendrag}{Sammendrag} % Bookmark name visible in a PDF viewer

\begingroup
\let\clearpage\relax
\let\cleardoublepage\relax
\let\cleardoublepage\relax

\chapter*{Sammendrag} % Abstract name

% Kort bakgrunn
% Hva er gjort
% Resultater. Referer de områdene dere har forbedret dere på, og hva dere har oppnådd.
% Vurderinger

Denne prosessrapporten ble skrevet i forbindelse med Eksperter i Team, landsby for Robot og Menneske. Arbeidet foregikk våren 2015. I denne rapporten ble det sett nærmere på situasjoner som oppstår når en tilfeldig sammensatt gruppe jobber sammen for å ta en idé fra utvikling til leveranse. Prosjektet gruppen jobbet med var en idé om å lage en autonom handlevogn. Meningen er at brukeren skal slippe å dytte rundt på vognen selv når han eller hun er ute og handler. Dette kan være spesielt nyttig i butikker med store og tunge gjenstander som møbelforhandlere eller byggvarehus. Det ble tatt i bruk en robot med sensorer for å detektere hindringer, i tillegg til et infrarødt kamerasystem som ble brukt til lokalisering av robot og menneske. I løpet av prosjektperioden har gruppen gjennomgått flere tester og undersøkelser, både som gruppe og enkeltindivider. I tillegg har gr\-uppen tatt opp situasjoner som har oppstått underveis, reflektert over de og kommet med aksjoner for å bedre samarbeidet. Noen har fungert, andre ikke. Selv om alle har jobbet med gruppearbeid tidligere, synes gruppen det har vært et nyttig emne og muligheten til å fokusere på selve gruppearbeidet har vært unik. Gruppen har lært mye om hvordan små justeringer i samarbeidet kan gjøre arbeidet mer effektivt, og om hvordan det er å jobbe i et team med ulik bakgrunn. Emnet har gjort gruppen mer reflektert, noe som mest sannsynlig vil hjelpe for å bli et best mulig gruppemedlem i fremtiden. 

\endgroup			

\vfill