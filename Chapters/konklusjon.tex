% Konklusjon

\chapter{Konklusjon} % Chapter title

\label{ch:konklusjon} % For referencing the chapter elsewhere, use \autoref{ch:mathtest}

%----------------------------------------------------------------------------------------

% Dere bør ha en konklusjon, bare for å runde av rapporten. Referer forbedringene deres kort som minimum, men gjør det kort hvis dere ikke har mer å konkludere med.

% Har ambisjonene og forventingene til faget blitt møtt? Lært noe? Hva eller ikke, og hvorfor / ikke. 

Jevnt over semesteret har vi hatt god stemning og et godt sam\-arbeid i gruppen. Vi har ikke hatt noen hendelser som har krevd store endringer i gruppen, men vi har gjort noen små justeringer der vi har sett et forbedringspotensiale. Vi ha blitt mer tydelig på morgenmøter om hva man har gjort og hva man skal gjøre, for å unngå at noe blir gjort to ganger. Vi har blitt flinkere til å ta ordentlig betenkningstid før viktige avgjørelser, slik at man ikke bare føler strømmen. Vi har blitt flinkere til å kun involvere dem i gruppen som trenger å bli involvert i en diskusjon, noe som frigir resten av gruppen til å jobbe videre med sitt arbeid. Vi har også blitt bedre på å sette en begrensing på avsporinger og tull.

Vi har alle i gruppen jobbet en del med gruppearbeid tidligere, men vi syntes likevel at emnet har vært nyttig. Eksperter i Team har gitt en mulighet til å fokusere på selve gruppearbeidet på en helt annen måte enn normalt gruppearbeid gjør. Normalt vil man ha så stort fokus på oppgaven som skal gjøres at man glemmer litt å tenke på hvordan man jobber sammen. Vi har igjennom emnet lært mer om hva som gjør en gruppe effektiv og hvordan vi kan gjøre justeringer på måten vi arbeider på for å bli mer effektive. Vi har fått erfare hvordan det er å jobbe samme i et team med andre som ikke har helt lik bakgrunn som en selv. Vi har også fått oppleve viktigheten med god kommunikasjon i en gruppe. Emnet har gjort oss mer reflekterte, noe vi tror vil gjøre oss til bedre gruppemedlemmer i fremtiden. 