% Bakgrunn og Teori

\chapter{Bakgrunn og Teori} % Chapter title

\label{ch:bakgrunnteori} % For referencing the chapter elsewhere, use \autoref{ch:mathtest}

%----------------------------------------------------------------------------------------

% Ingen eller flere kapitler med teori som det er hensiktsmessig å samle her for senere å kunne referere lett til. Ikke legg arbeid i å referere EiT ``standardpensumet", men alt dere finner selv og som er relevant for dere, er interessant å ta med. \todo{Endre eller fjern innledning til teori}

I dette kapittelet presenteres bakgrunn og teori for resultatene som kommer frem senere i rapporten. 

\section{Personlighetstest Myers-Briggs}


<<Myers-Briggs Type Indicator>>\footnote{\url{http://www.16personalities.com/free-personality-test}} er en personlighetstest som inneholder 4 par av dimensjoner, hvor hvert par representerer motsetninger. Resultatet på testen blir en bokstav-kombinasjon bestående av 4 av disse dimensjonene, gitt av de engelske ordene for de ulike dimensjonene \cite{16personalities} \cite{MyersBriggs}. Disse er:

\begin{center}
\begin{tabular}{l l | l l}
I: & Introversion & E: & Extroversion \\
N: & Intuition    & S: & Sensing \\
F: & Feeling      & T: & Thinking \\
P: & Percevering  & J: & Judging \\
\end{tabular}
\end{center}


\subsection{Introvert og ekstrovert}

Introverte mennesker velger å fokusere mest på sin egne indre verden av refleksjoner, følelser og tanker. De har ingenting imot å jobbe alene, og klarer å ta avgjørelser uavhengig av andre. Introverte mennesker blir tappet for energi i sosiale sammenhenger og trenger tid alene for å ta seg inn igjen. 

Ekstroverte mennesker er motpolen til introverte mennesker. De velger å rette fokuset mot den ytre verden. De liker aktivitet, er sosiale og pratsomme, og liker å omgås med andre mennesker. Ekstroverte mennesker får energi av å omgåes andre mennesker og har mindre behov for å ha tid alene enn hva en introvert har.

\subsection{Intuisjon og sansing}

Intuitive mennesker kjeder seg over rene fakta og detaljer, de liker å ha idéer, muligheter og teorier. 

Sansende mennesker er realistiske, praktiske og jordnære. De liker det konkrete, og de benytter sansene sine til å forstå verden omkring seg. Vanligvis er de tålmodige, rutinerte og nøyaktige.

\subsection{Følende og tenkende}

Følelsesmennesker tar avgjørelser basert på følelser og personlige verdier, de er opptatt av harmoni og samarbeid. De er sosiale og interessert i mennesker, og kommer derfor som regel godt overens med andre.

Tenkere er logiske, rasjonelle og analytiske. De tar avgjørelser basert på grunnlag av verifiserbare konklusjoner, og er ikke så mye påvirket av følelser og verdier.

\subsection{Oppfattende og dømmende}

Oppfattende mennesker er mer opptatt av veien til målet, enn selve målet. De liker å holde muligheter åpne, de er gode på å improvisere og har et mer avslappet forhold til arbeid.

Dømmende mennesker er mer opptatt av det endelige resultatet, enn veien dit. De liker å ha en klar struktur i hverdagen og er mest komfortable med klare regler og retningslinjer å forholde seg til.

\section{Gruppens personligheter}

I gruppen har vi, etter Myers-Briggs personlighetstest, følgene personlighettyper:
\begin{center}
\begin{tabular}{l c r}
ENFJ\\
ENFP\\
ISFJ\\
ISTJ (tre stykker)\\
\end{tabular}
\end{center}

\section{Samarbeidsindikatoren}
Samarbeidsindikatoren er en undersøkelse som har til hensikt å gi en gruppe et inntrykk av hvordan samarbeidet mellom dem fungere. Undersøkelsen består av at alle gruppemedlemmene skal krysse av på i hvor stor grad en rekke utsagn passer for gruppen. Utsagnene har rot i fire forskjellige faktorer:
\begin{itemize}
\item Åpen og lyttende
\item Ærlig og direkte
\item Forpliktet og tillitsfull
\item Effektiv og strukturert
\end{itemize}
Resultatet av undersøkelsen vil, ut i fra gruppens svar, vise i hvilken grad disse faktorene ble nådd. Skalaen på undersøkelsen går fra 0 til 6 hvor 0 er <<ikke i det hele tatt>> og 6 er <<i svært stor grad>>. 

\section{Roller}
Roller er en gruppeøvelse utarbeidet av Are Holen som har til hensikt å bevisstgjøre gruppemedlemmer på hvilke rolle dem og de andre i gruppen har. Øvelsen går ut på at hvert gruppemedlem skal gi poeng mellom 0 og 9 til alle i gruppen, inkludert seg selv, ut ifra hvor godt forskjellige utsagn passer til personen.
De ulike utsagnene var:
\begin{enumerate}
\item Tar ledelse, setter mye preg på gruppens retning og aktivitet
\item Innordner seg, er underkastende, taus eller stille
\item Er handlingsorientert, innstilt på å få ting gjort, produksjons\-rettet 
\item Er sosial, bidrar til å skape hygge og trivsel
\item Opptar mye av gruppens oppmerksomhet (enten du liker det eller ikke)
\item Er lite villig til å følge felles opplegg, vil ha ting gjort på sin egen måte, kan være i opposisjon, bli borte eller ha sin oppmerksomhet andre steder
\end{enumerate}


