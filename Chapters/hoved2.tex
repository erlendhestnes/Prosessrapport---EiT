% Hoveddel

\chapter{Hoved2} % Chapter title

\label{ch:hoved2} % For referencing the chapter elsewhere, use \autoref{ch:mathtest}

%----------------------------------------------------------------------------------------

% Skrives i fortid.

% Gjengi situasjoner med forbedringspotensial. 
% Referanse til teori som kaster lys over utfordringen. 
% Reflektere over svakheter og hvordan vi kan forbedre oss, kombinert med teori. 
% Evaluere om målene ble nådd.


Alle problemer som oppsto iløpet av tiden gruppen har jobbet sammen valgte vi å løse med en metode fra en bok av  Susan Wheelan \cite[s.
61-63]{wheelan_creating_2012} som går ut på å gjenkjenne problemet, analysere problemet, ta en avgjørelse og akseptere og følge denne avgjørelsen. Dette var en metode som fungere bra for vår gruppe. I hoveddelen har vi valgt å fokusere på to hovedtema; idéavklaring og effektivitet.

\section{Idéavklaring}
Problemene til gruppen vår startet allerede i idéfasen, hvor problemstillingen skulle bestemmes. Her har gruppen vår hatt mye motgang. En problemstilling ble ikke godkjent av landsbyleder og en problemstilling samt enkelte tekniske løsninger har vi måtte forkastet pga. lite gjennomførbarhet.

Når gruppen ikke fikk godkjent første problemstillingen sin var dette fordi landsbylederen mente idéen ikke var relevant nok til landsbytemaet <<Robot og Menneske>>. De fleste på gruppen var ikke forberedt på dette i det hele tatt, og ble derfor ganske skuffet siden vi allerede hadde brukt mye tid på å forme idéen og gjort oss opp en del tanker rundt løsninger og arbeidsfordeling. To av gruppemedlemmene hadde vært litt mer forberedt på at vi kanskje måtte bytte problemstilling, og ble derfor ikke like skuffet. De hjalp resten av gruppen med å omstille seg og starte idémyldringen igjen.  

Etter at ny problemstilling var bestemt oppsto det et nytt problem. En i gruppen hadde brukt litt tid mellom to landsbydager til å finne ut mer om teknologien gruppen hadde planlagt å bruke. Han fant da ut at teknologien ikke lot seg bruke til vårt formål og han var derfor allerede klar for å bytte problemstilling når han kom til neste landsbydag. Etter litt diskusjon, for å få litt fortgang på arbeidet, la han litt press på gruppen til å ta et valg mellom et par ny problemstillinger vi hadde foreslått. Flere på gruppen følte ikke de hadde fått tid til å tenke seg godt nok om, en av dem sa ifra om dette. \todo{aksjon}Dermed tok gruppen en god pause og vi fikk mer tid til å tenke. Denne betenkningstiden gjorde at vi endte opp med en helt annen problemstilling, enn hva vi egentlig skulle ha valgt mellom tidligere. Denne nye problemstillingen ble grunnlaget for prosjektet sånn det endte opp. Etter at gruppen fikk oppleve hvor stor forskjell ordentlig betenkningstid kan gjøre, ble vi veldig oppmerksom på å ha nok tid før viktige avgjørelser. Om vi visste at en avgjørelse måtte bli tatt ved neste landsbydag gjorde vi det til en vane å ta opp saken på slutten av dagen, slik at vi kunne tenke på det til neste gang.

Gruppen måtte flere ganger ta stilling til hvilke tekniske løsninger vi skulle gå for å realisere prosjektet vårt. Vi måtte blant annet avgjøre om vi skulle bruke kamerasystemet eller ikke. For at vi skulle få tatt en avgjørelse uten at alle i gruppen brukte tid på å sette seg inn i begge alternativene, \todo{Aksjon?}lot vi ene gruppemedlemmet, som hadde litt mer oversikt enn andre, presenterte både for- og motargumenter for begge alternativene. Selv om han hadde gjort seg opp en mening allerede, la han frem argumentene på en objektiv måte og forsikret gruppen om at han ville respektere flertallets avgjørelse uansett. Resten av gruppen syntes dette var en veldig ryddig og fin måte å gjøre det på og gjorde det lettere å ta en avgjørelse.

\section{Effektivitet}

Allerede fra første dag har det vært et fokus på å holde høy effektivitet innad i gruppen, slik at jobbing utenom onsdager ble ungått. Dette ble også formalisert som et eget punkt i kontrakten som gruppen skrev i starten av perioden. I starten var det veldig mye diskusjon, noe som ofte førte til avsporinger og tulling. Dette var noe vi tidlig oppdaget gikk ut over effektiviteten i gruppen. Som en aksjon til dette problemet begynte vi å sette tidsfrister på useriøs diskusjon. Når et medlem på gruppen merket at diskusjonen begynte og bli useriøs kunne han eller hun si ifra om at dette fortsetter vi med i noen minutter til, før vi går tilbake til tema. Dette var en avgjørelse som alle var enige i og den ble dermed tatt i bruk. Litt avsporing og tulling bidrar til god gruppefølelse og gir rom for en mer avslappet atmosfære. Derfor var det ikke utelukkende negativt at dette fant sted, spesielt ikke i startfasen. 

Noen medlemmer av gruppen hadde erfaring fra arbeidslivet, og en del effektiviseringsteknikker ble derfor hentet derfra. I begynnelsen ble blant annet Scrum \cite{scrum} tatt i bruk. \todo[nolist]{Aksjon} Såkalte <<Stand-up>>-møter har også blitt benyttet av gruppen. Dette er korte morgenmøter hvor hvert medlem forteller om hva som ble gjort forrige gang, og hva som skal gjøres for dagen. Morgenmøtene har bidratt til å iverksette medlemmene på gruppen, og effektivt sørget for at hvert medlem til enhver tid har hatt noe å gjøre.

Gruppen har til tider hatt problemer med dårlig kommunikasjon. Dette har ført til at det har blitt brukt mye tid på arbeid som senere har vist seg og ikke være direkte nyttig for resultatet. For eksempel så brukte flere i gruppen mye tid på å sette seg inn i ulike stifinnings\-algoritmer som roboten skulle bruke. På et senere tidspunkt så oppdaget enkelte på gruppen at dette allerede eksisterte inne i programvare-rammeverket til roboten. Denne situasjonen kunne ha vært unngått om vi hadde en person som bedre kjente til alle funksjonene til roboten. Selve hendelsen førte til tapt tid for gruppen og nedsatt motivasjon for dem som hadde brukt tid på arbeid som det ikke var behov for likevel. 

En av hovedårsakene til nedsatt effektivitet i gruppen var at kun én datamaskin kunne kommunisere med roboten. Dette begrenset mengden med teknisk arbeid som kunne utføres samtidig. \todo{Mulig aksjon om vi hadde hatt bedre tid}Det finnes et simuleringsverktøy for roboten som kunne ha fjernet denne flaskehalsen. Hadde dette verktøyet blitt brukt, så ville helt klart effektiviteten ha økt på sikt, men man ville brukt mye tid på å sette seg inn i dette. Med tanke på at prosjektetperioden var så kort ville det trolig ikke vært lønnsomt i vårt tilfelle. Gruppen kom derfor frem til at simuleringsverktøyet skulle nedprioriteres.

I løpet av arbeidsperioden som gruppen har vært gjennom, har det vært tydelig at engasjement og motivasjon er en viktig faktor for effektivitet. En kunne observere en klar nedgang i effektiviteten da gruppen ble nødt til å endre kurs som en følge av råd fra veileder. 



