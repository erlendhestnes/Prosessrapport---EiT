% Innledning

\chapter{Innledning} % Chapter title

\label{ch:innledning} % For referencing the chapter elsewhere, use \autoref{ch:mathtest}

%----------------------------------------------------------------------------------------

% Ikke føl noen plikt til å gjøre denne lang, men ha:
% - Innledning
% - En kort presentasjon av gruppemedlemmene (Hvem er du, Bakgrunn, Innstilling til faget før begynnelsen, Forventning til faget, Motivasjon, Ambisjon) 
% - Eventuelt et veldig kort sammendrag av fagrapporten/prosjektet er flott.
% - I tillegg: Ting dere føler passer her. Spenn leserens forventninger.

% Skrives i nåtid. 

Hvilken påvirkning har adferden til et gruppemedlem på de øvrige i en gruppe, og hvordan påvirkes adferden til gruppemedlemmet av de andre? Hvilke prestasjoner, trender, handlinger eller egenskaper ved et gruppemedlem har en stimulerende effekt på gruppearbeidet, og finnes det et system i gruppen for å videreføre disse? Dette er noen av aspektene ved gruppearbeidet som vil granskes i denne rapporten. I tillegg vil det legges vekt på egne erfaringer og refleksjoner som er gjort.  

Denne prosessrapporten er skrevet i forbindelse med gruppe 4 sitt arbeid i emnet TTK4851 Eksperter i Team, landsby for Robot og Menneske. Arbeidet foregikk våren 2015, fra oppstart 7. januar, til levering 29. april. I prosessrapporten sees det nærmere på situasjoner som oppstår når en tilfeldig sammensatt gruppe, med en moderat grad av tverrfaglighet, jobber sammen for å ta en idé fra utvikling til leveranse. 

For å unngå å være personlige brukes det ikke navn når det er snakk om personer i denne rapporten, med unntak av introduksjon av medlemmer eller personlig refleksjoner. Ved tilfeller som gjelder undersøkelser eller situasjoner i gruppen blir personer kun referert til som <<Person 1>> eller lignende. 

\section{Gruppesammensetning}

Gruppen består av seks medlemmer, hvor fem av disse er gutter. Alle studerer teknologirelaterte studier, hovedsaklig fordelt på kyberne\-tikk, informatikk, elektronikk og undervannsteknologi. 

Jon Zwaig Kolstad er en 23 år gammel mann, opprinnelig fra Hønefoss med en bachelorgrad som elektronikk- og IT-ingeniør fra Høgskolen i Oslo og Akershus. Han studerer for tiden master i Teknisk Kybernetikk og tar Eksperter i Team med innstilling om å lære mer om hvilke(n) rolle(r) han tar i en gruppe, og fokuserer i hovedsak på denne kunnskapen fremfor kunnskapen han tilegner seg om det tekniske temaet. 

Harald Moholt er 23 år gammel og kommer fra Bærum. Utdannet med bachelorgrad innen Automatiseringsteknikk fra Høgskolen i Ålesund før han begynte på toårig master i Undervannsteknologi på NTNU. Under Eksperter i Team er han forberedt på å lære mye om seg selv i en gruppe i tillegg til å utforske muligheter rundt roboter i samspill med mennesker i samfunnet. Motivasjon er å lære mer om roboter i samfunnet, og hvordan de kan brukes eller utnyttes til menneskes beste. Harald har ambisjon om å lage noe praktisk som viser en eller flere idéer om hvordan roboter kan brukes i dag.

Erlend Hestnes er 22 år gammel og kommer fra Trondheim. Han går for tiden 4. året på Elektronikk, NTNU. Hans forventninger til Eksperter i Team er først og fremst å lære mer om seg selv. Han ønsker blant annet å få innsikt i om hans personlige oppfattelse stemmer overens med andres. EiT landsbyen, Robotikk og Menneske, er knyttet tett opp imot hans egne interesser. Han ser derfor frem til å jobbe sammen med likesinnede eksperter om å få til noe stort. 

Håvard er en 26 år gammel mann fra Skogn. Har gjennomført en treårig bachelorgrad som dataingeniør ved Høyskolen i Sør-Trønde\-lag før han begynte på toårig master utdanning i Informatikk ved NTNU. Han er innstilt på at Eksperter i team skal bli gøy og at det er en gunstig mulighet for å utvikle seg personlig. Han håper det blir gøy og at han får jobbe med et spennende tema. Han er motivert med tanke på teambuilding, og ønsker å se hvordan han passer inn i en helt fremmed gruppe. Vil også se hvordan egenskapene hans passer med de andres, og ønsker å få til et godt samarbeid og gjerne finne ut om hans oppfatning av seg selv stemmer overens med andres oppfatning.

Helene Myre er 24 år og kommer fra Bergen. Hun går i 4. klasse på Teknisk Kybernetikk, NTNU. Hun ser frem til jobbe med emnet og håper å forbedre sine evner til å tilpasse seg ulike gruppedynamikker, slik at gruppearbeid i fremtiden blir så effektivt som mulig. Hun er også motivert av tanken på å kanskje få være med å utvikle et spennende produkt.

Morten Mjelva er 29 år og kommer fra Trondheim. Han går i 5. klasse på Teknisk Kybernetikk, NTNU. Han ser frem til å lære mer om dynamikken og rollefordelingen i en så stor arbeidsgruppe, og er spent på hvilken rolle han selv vil innta. Han ser frem til å lære mer om robotikk, og hvordan robotikk kan påvirke menneskers hverdag, i løpet av emnet.

\section{Prosjektet}

Selve prosjektdelen av emnet har bestått av å utvikle en robot som følger etter en bruker. Til dette ble det brukt en ferdig robot, Pioneer P3-DX, som instituttet hadde tilgang på. Det ble i tillegg brukt et avansert kamerasystem som er fastmontert på et laboratorium. Roboten har hjul og diverse sensorer som ultralyd, støtfangere og laser-scanner. Kamerasystemet består av 16 infrarøde kameraer som kan oppfatte spesielle objekter. Prosjektet bestod hovedsaklig av å utnytte egenskapene ved roboten og kamerasystemet til å lage en løsning på et scenario hvor roboten forfølger en bruker i et smart mønster samtidig som den unngår kollisjon med hindringer. 

\section{Inndeling av rapporten}
\paragraph{Bakgrunn og teori}
I dette kapittelet presenteres den nødven\-dige bakgrunnsinformasjonen og teorien som blir brukt i rapporten. 

\paragraph{Hoveddel}
Formålet med dette kapittelet er å vise gruppens utvikling gjennom forskjellige situasjoner eller hendelser. Her fokuseres det på spesifikke aspekter ved samarbeidet. 

\paragraph{Resultater}
Her presenteres de resultatene som har kommet frem gjennom undersøkelser eller tester i løpet av prosjektperioden. Det inneholder også personlig refleksjon til alle gruppemedlemmene. 

\paragraph{Diskusjon}
Dette kapittelet forteller mye om hvordan gruppen ser tilbake på arbeidet og fokuserer på positive eller negative valg som kunne blitt gjort annerledes. 

\paragraph{Konklusjon}
Dette runder av rapporten og konkluderer eventuelle meninger eller diskusjoner som ble gjort underveis. 