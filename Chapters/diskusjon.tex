% Diskusjon

\chapter{Diskusjon} % Chapter title

\label{ch:diskusjon} % For referencing the chapter elsewhere, use \autoref{ch:mathtest}

%----------------------------------------------------------------------------------------

% Skrives i fortid.

% Bare hvis dere har noe mer å diskutere, men det tar seg alltid godt ut å være kritisk til eget arbeide her.

% Burde kanskje ha dratt inn pensum i arbeidet.

% Hvorfor er dette faget bra?
% - Eksempel med rollebevissthet i jobbintervju f.eks.


Gruppen brukte mye tid i starten på å finne en problemstilling som engasjerte alle gruppens medlemmer og hvor alle følte de fikk brukt sin kompetanse. Både første og andre idé ble avslått av ulike årsaker. Dette førte til at gruppen begynte å føle et tidspress for å komme igang med arbeidet og at problemstillingen vi endte opp med ble fattet på et litt dårligere grunnlag enn hva de foregående problemstillingene hadde blitt. Alle i gruppen var engasjerte for idéen, men kompetansen til gruppen ble ikke like godt ivaretatt. 

Når arbeidet startet tok et av gruppemedlemmene fort styringen på det tekniske. Dette gruppemedlemmet fikk naturlig nok mest kunnskaper om systemet vårt og for å spare mest mulig tid ble det han som jobbet videre med det tekniske hver gang, mens resten tok for seg andre oppgaver. Dette var en avgjørelse gruppen i ettertid har diskutert og kommet frem til at var en dårlig løsning. Selv om det, der og da, var mest effektivt og la den med mest kunnskaper gjøre jobben, så gjorde vi oss selv veldig sårbare som gruppe ved å være så avhengig av én person. Det kom frem at dette var noe de fleste i gruppen hadde tenkt på uten å si noe, men innen vi tok det opp i gruppen, var det gått såpass lang tid at å lære opp andre gruppemedlemmer i programvaren ville ta for lang tid. Vi hadde alle sammen hatt en litt for passiv holdning til problemet. Dette kan ha noe med at ikke alle på gruppen følte de hadde like mye å bidra med på det tekniske etter at vi måtte bytte problemstilling. 

Som gruppe burde vi nok kanskje stoppet litt mer opp før den endelige problemstillingen ble bestemt. Vi kunne da kanskje innsett tidsnok at kompetansen som skulle til for å løse oppgaven var mer skjevt fordelt blant gruppemedlemmene med denne problemstillingen enn med de tidligere forslagene. Vi kunne også vært flinkere til å si rett ut at vi syntes arbeidsfordelingen ikke var helt som ønsket, slik at vi kunne vært flere på den tekniske delen fra starten. 

Gruppen ble tidlig i idemyldringsfasen opphengt i at resultatet skulle bli noe som fungerte i praksis. Denne tanken var noe som motiverte alle og vi så derfor på dette som viktig å få gjennomført. Vi kunne kanskje valgt å ikke fokusere så sterkt på dette, noe som kunne ført til at vi hadde kommet opp med flere ideer som lot seg gjennomføre på en mer effektiv måte enn den vi endte opp med. 

Det at gruppen har vært ganske faglig homogen kan ha vært med å påvirke gruppearbeidet vårt. I David Johnson og Frank Johnson sin bok om grupper, i kapittelet <<Valuing diversity>> \cite[s. 452-453]{johnson_joining}, blir det nevnt flere ulemper med homogene grupper. Uenigheter og motstridende perspektiv i en gruppe er ofte en god måte å komme frem til best mulig avgjørelser og for å fremme kreativ tenkning. I en homogen gruppe risikerer man at gruppemedlemmene tenker for likt og dermed går glipp av spennende innspill man ellers ville fått. Homogene grupper unngår oftere å ta sjanser og går dermed glipp av muligheter til å øke produktiviteten. De inngår oftere i gruppetenkning \cite{groupthink}, noe som kan føre til dårlige avgjørelser. Siste som ble nevnt i teksten var at homogene grupper fungerer best i statiske situasjoner og har større vanskeligheter for å tilpasse seg endringer. Det er naturlig nok vanskelig å vite hvordan vi ville vært som gruppe om vi hadde hatt et større faglig mangfold, men vi syns det er grunnlag for å tro at en del kunne vært anderledes. Vi hadde nok vært mindre ensporet under idemyldringen og generelt fått flere friske og nytenkende innspill i diskusjonene våre, noe som fort kunne ført til andre avgjørelser enn hva vi endte opp med. Vi hadde til tider litt problemer med å omstille oss når vi f.eks. måtte bytte problemstilling. Hadde vi hatt et større mangfold på gruppen kunne det kanskje ha hjulpet oss til å raskere komme oss videre. Vi føler ikke at mangel på faglig mangfold har ført til dårlige avgjørelser, men vi tror større mangfold kunne fått oss til å bli mer nytenkende og til å ta flere sjanser. 

